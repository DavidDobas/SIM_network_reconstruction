\chapter*{Conclusion} % SEM NESAHEJTE!
\addcontentsline{toc}{chapter}{Conclusion} % SEM NESAHEJTE!
\markboth{Conclusion}{Conclusion}
In this research projected, we introduced basic network models as well as the novel Scale-invariant model, outlined the problem of network reconstruction with a focus on financial networks and then applied the Scale-invariant model in the reconstruction problem. We encountered a significant problem using the original Scale-invariant model, where the model would produce samples with half of nodes ending up isolated on average. Since this would largely influence study of dynamical processes on such reconstructed networks, we proposed o modification of the original model and a sampling strategy, which can avoid the production of isolated nodes. We studied other properties of the corrected model compared to the original version and found out, that although the problem of isolated nodes was solved, the quality of reconstruction did not largely improve when studying other network properties.

Although we came with some novel ideas, several aspects still need to be studied to fully explore the strength of the Scale-invariant model in the problem of network reconstruction. First, we shall note that not every model is useful in every situation. Our primary aim was to use the model in the area of financial networks, however, the datasets are difficult to obtain. We therefore used the airport network data, which, as we noted, might have different properties than the financial networks. The first important step would be to study the model on the true networks of interest and ideally even recognize the range of applicability.

Secondly, we made several hypotheses, for example that in some situations the predictions are not precise enough because of the limited number of samples in our ensembles or that the Metropolis-Hastings algorithm did not yet reach the equilibrium configurations of networks. These hypotheses need to be proved true or wrong, which would, however, need larger computational capacity. 

Another important step would be to not only study new network properties as we did, but also the properties of dynamical processes, like DebtRank. Only by that, we can prove the usefulness of our method, if we can for example predict the stress propagation on the network, whose structure we do not fully know and which we need to first reconstruct. 

Last but not least, we shall recall that there already exist several network reconstruction methods, and we should therefore compare their performance with our method. Our first motivation was, that the Scale-invariant model may work better in situations, where multiple scales are present, however, we did not evaluate that hypothesis yet. Also, there exist quantitative methods to compare the quality of reconstruction methods (as mentioned for example in \cite*{Squartini2018}), which shall be used to show the quality of our method. 

To conclude, in this project, we only started research in a topic, which opened many interesting questions to be further investigated. We can not conclude yet whether the proposed method brings significant improvements over already existing ones. On the other hand, we proposed an approach to avoid the creation of isolated nodes, which can be used not only in our method. 