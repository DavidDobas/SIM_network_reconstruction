\chapter*{Introduction} % SEM NESAHEJTE!
\addcontentsline{toc}{chapter}{Introduction} % SEM NESAHEJTE!
\markboth{Introduction}{Introduction}
In the recent years, a big portion of interest in the field of complex systems has started to be devoted to the model of complex networks. Networks allow for the study of complex systems as comprised of elements, which are called nodes, and pairwise interactions, called edges. Many such systems have been then studied, and several observations have been made, such as that real-world networks are usually scale-free (in terms of degree distribution, as explained later), that they manifest a high level of clustering, or have small distances on average. These observations started a wave of interest in the ways in which networks and their creation can be modeled mathematically. 

Several models have been proposed, some of which we also introduce in this thesis. Recently, the problem of different scales in networks has started to draw interest. In this work, we introduce the recently proposed Scale-invariant model, which allows for modelling of networks consistently across all possible scales, while not relying on any geometrical embedding of the network in study.

Another interesting topic has emerged in the recent years, which is the problem of network reconstruction. The most general setting is that we have any partial information about the network of interest, and we want to find the closest reconstruction of such a network using only that partial information. A specific case of financial networks is a highly important one, since one can use the reconstruction methods to subsequently study stress propagation in such networks and the possibility of crises and crashes. 

In this thesis, we aim to briefly introduce the setting of complex network science and then focus in more detail on the Scale-invariant model and the network reconstruction problem. Then we combine these ideas together and use the Scale-invariant model in network reconstruction. We show several properties of the proposed method and tackle problems that might emerge along the way. 